\documentclass[a4paper]{article}

\RequirePackage[utf8]{inputenc}		% remove when using lualatex oder xelatex!
\RequirePackage{xcolor}
\RequirePackage{comment}
\RequirePackage{verbatim}
\RequirePackage{amsmath}
\author{Wilhelm Burger\\ 
\texttt{wilbur@ieee.org}}

\title{How LaTeX Handles White Space and Newlines}
\date{}

%%%----------------------------------------------------------
\begin{document}
%%%----------------------------------------------------------
\maketitle
%%%----------------------------------------------------------
\clearpage
\tableofcontents
\clearpage
%%%----------------------------------------------------------

\begin{abstract}\noindent
This document shows the handling of white space and new/blank lines, as
well as LaTeX's behavior in various other situations.
\end{abstract}

\section{Whitespace and Newlines}

These are test cases used in the parser test (\verb!Test_MULTISTRING!).
Empty lines contain nothing or only horizontal whitespace.

\paragraph{A. A single newline is treated as whitespace:} ~\\
\verb!"a simple   newline\nin the text"!\\
a simple   newline
in the text

\paragraph{B. A pair of newlines makes a paragraph:} ~\\
\verb!"a pair of new lines\n\nmakes a paragraph"!\\
a pair of new lines

makes a paragraph


\paragraph{C. Two or more successive empty lines makes a paragraph (same as above):} ~\\
\verb!"a pair of \n\t \n empty lines"!\\
a pair of 
	
 empty lines
	
\paragraph{D. blank lines with a trailing comment are not considered empty:} ~\\
\verb!"a blank line\n   %comment\nwith a trailing comment"!\\
a blank line
   %comment
with a trailing comment

\section{What's allowed in textmode}

\begin{enumerate}
\item
	Are underscores allowed in \verb!text_mode!? NO (math mode only)!
\item
	Here is an accented character in foo\`{o}zzle or ba\c is.
\item
	What happens to a standalone \verb!\-! (optional hyphen) like \- here? 
	It inserts extra space, no error!
	Even multiple insertions \-\-\-\- are OK.
\end{enumerate}

\section{Comments and Newlines}

\noindent\textbf{Case 1:}
Everything after a trailing {\color{blue}comment%comment
    IS} ignored, even if followed by a line with leading blanks (or tabs) -- 
which are \textbf{all ignored}.
This means that ALL leading white space must be ignored too!
Note that the newline after the comment is NOT considered white space!
\begin{verbatim*}
Everything after a trailing {\color{blue}comment%comment
    IS} ignored, even if followed by a line with leading blanks 
(or tabs) -- which are \textbf{all ignored}.
\end{verbatim*}

\bigskip\noindent\textbf{Case 2:}	
A line which starts with a \%-sign is {\color{blue}ignored,
		%
even} if it contains leading whitespace. It is not treated as an empty line (paragraph).
Note that the \texttt{newline} after ``ignored,'' IS considered white space!
\begin{verbatim*}
A line which starts with a \%-sign is {\color{blue}ignored,
      %
even} if it contains leading whitespace.
\end{verbatim*}

\bigskip\noindent\textbf{Case 3:}
One or more empty lines are {\color{blue}treated as
    
new} paragraphs, even if they contain white space (but nothing else).
\begin{verbatim*}
One or more empty lines are {\color{blue}treated as
    
new} paragraphs, even if they contain white space ...
\end{verbatim*}



\bigskip\noindent\textbf{Case 4:}
If a line ends with a {\color{blue}comment%
      
then} the subsequent empty line(s) is still treated as a paragraph.
Interesting: {\color{red}braces may stretch over paragraphs!}
\begin{verbatim*}
If a line ends with a {\color{blue}comment%
      
then} the subsequent empty line(s) is still treated as a paragraph.
\end{verbatim*}



\bigskip\noindent\textbf{Case 5:}
Multiple lines with {\color{blue}comments%
   %
%
% anything
		%    anything
are} treated as a single, contiguous comment!
\begin{verbatim*}
Multiple lines with {\color{blue}comments%
   %
%
% anything
		%    anything
are} treated as a single, contiguous comment!
\end{verbatim*}



\section{Whitespace with commands}

\noindent\textbf{Case 1:}
Whitespace between the command and its \textbf {argument} does not matter,
as in \verb*!\textbf {argument}!.
Even a \emph{single} \textbf
{newline}
is OK before the argument.


\bigskip\noindent\textbf{Case 2:}
Leading whitespace in a command-\textbf{ argument} is \emph{not} ignored,
as in \verb*!command-\textbf{ argument}!.

\bigskip\noindent\textbf{Case 3:}
Some commands like \thesection      
do consume all subsequent whitespace, such as \verb*!\thesection   !.


\section{Whitespace with environments}

\noindent\textbf{Case 1:}
If we leave a blank after \verb*!\begin!, as in \begin   {tiny}this is tiny text\end{tiny} 
this does not matter.
Even a line break after \verb*!\begin! \begin   %
  %
{tiny}is allowed.\end{tiny}.

\bigskip\noindent\textbf{Case 2:}
However, it is an \textbf{error} to write \verb*!\begin{ tiny}! or \verb*!\begin{tiny }!
instead of \verb!\begin{tiny}!.



\section{Special characters}

\subsection{Reserved characters}

Characters not allowed in normal text are  
\begin{itemize}\itemsep0pt
\item[\texttt{\%}] starts a comment,
\item[\texttt{\$}] starts inline math,
\item[\texttt{\#}] only for macro parameters,
\item[\texttt{\_}] underscore only allowed in math mode,
\item[\texttt{\&}] only as alignment tab in tables.
\end{itemize}
%
These are OK: @.


\subsection{Escaped characters}

These are \%, \$, \#, \_, \& (\verb!\%, \$, \#, \_, \&!).



\subsection{Parentheses and brackets}

Parentheses (in plain text) ave no special meaning.
Unbalanced (parentheses] do [no harm).
Pairs of brackets [alone] are permitted everywhere
(not only as optional command arguments).


%%%----------------------------------------------------------

\section{Labels}
\label{sec:this 123}
\label{a]bra}

\noindent\textbf{Case 1:}
Labels such as ``\verb!sec:this 123!'' are legal and can be properly referenced (\ref{sec:this 123}).
Thus digits and blanks are allowed, but also 
\verb*" :;,.!_$&/()[]=?*+-@^|"
and even
\verb*!"'!. The latter should be discouraged!
%(\ref{a]bra})

\bigskip\noindent\textbf{Case 2:}
Umlauts (and probably most other UTF-characters) are not allowed in labels.



%%%----------------------------------------------------------

\section{Spacing}

\begin{itemize}
\item Text: Here we{\thinspace}use\,thinspace (\verb!\thinspace! \verb!\,!).
\item Text: Here we\ use regular space (\verb*!\ !).
It also works with a\
newline (i.e., if the \verb"\" is at the end of the line)!
\item Text: Here we use~tilde (\verb!~!).
\item Text: Here we use\enspace enspace (\verb!\enspace!).


\item Math space (thin): $a\,b$ (\verb!\,!).
\item Math space (med): $a\:b$ (\verb!\:! or \verb!\>!).
\item Math space (thick): $a\;b$ (\verb!\;!).
\end{itemize}

\noindent
Are these the same?:
\\
Spacing at a newline Fig.\
3.10 (newline).
\\
Spacing at a newline Fig.\ 3.10 (no newline).
\\
Spacing at a newline Fig.\	3.10 (w.\ tab).
\\
Spacing at a newline Fig. 3.10 (without \verb*!\ !).


%%%----------------------------------------------------------

\section{Manual line breaks}

\noindent
In this text we break the line\\
at this point.

\noindent
In this text we break the line\\*
at this point.


\noindent
In this text we break the line\linebreak
at this point.

\noindent
In this text we break the line\break
at this point.

%%%----------------------------------------------------------

\newcommand{\foo}[2][default]{Mandatory: #2; optional: #1}

\section[Ze Command Section]{Commands}

This section entry carries an optional argument for the TOC (\verb![Ze Command Section]!):
\begin{itemize}
\item
LaTeX appears to read optional arguments by only scanning for the next-following \verb!]!.
\item
Thus no nesting of \verb![..]! is allowed inside, otherwise LaTeX gets confused (no error though).
\item
Blocks \verb!{..}! are allowed.
\item
Commands are allowed, but none with optional parameters in \verb![..]!
\item
Inline math is allowed.
\end{itemize}


%%%----------------------------------------------------------

\section{Environments}

Unbalanced environments should confuse LaTeX. The following two examples should work properly:

\begin{center}
outer text
\begin{quote}
inner text
\end{quote}
\end{center}


\begin{quote}
outer text
\begin{center}
inner text
\end{center}
\end{quote}

But this gives an error:
\begin{verbatim}
  \begin{quote}
     outer text
     \begin{center}
      inner text
    \end{quote}
  \end{center}
\end{verbatim}





%%%----------------------------------------------------------

\section{Comment environment}

The \verb!\begin{comment}! may be placed anywhere in a line and
\verb!\end{comment}! as well.
Note that the rest of the line following \verb!\end{comment}! is also ignored!
%
\begin{verbatim}
\begin{comment}
some comment text ...
  \end{comment}  this is IGNORED
\end{verbatim}
%
Thus there should be only a blank space between here: \begin{comment} some comment text ...
   \end{comment}  this is ignored
and the following text.

\noindent
This single-line comment environments are valid too with the same rules as above:
%
\begin{verbatim}
\begin{comment} some comment text ... \end{comment} IGNORED!
\end{verbatim}

%\begin{comment}   \end{comment}
%%\begin{comment}
%some comments ...
%%\end{comment}
%\end{comment}

\section{Verbatim text}

\subsection{Inline verbatim with \texttt{verb}}

%This is regular \verb!verbatim text! without any newlines in it.
\verb!\verb! must not contain an end of line!
The delimiting character following \verb!\verb! may be any non-character, including 
`\texttt{*}' (even blank is allowed!).
Thus \verb!\verb**\foo*! works!

The delimiting character may be a number, as in \verb!\verb6verbatim text6!.
Underscores \verb_also work_, as do \verb"double quotes" and \verb'single quotes',
even \verb!%! signs.

\verb!\verb*! makes readable blank spaces:
as in \verb*+this small example+. It does accept \texttt{*} as a delimiting character.

If \verb!\verb! runs to the end of the current line!
but is not properly terminated, this results in an error!

If \verb!\verb! is followed by a \% symbol
   then the following newline (and following whitespace) is ignored (as usual)!



\subsection{Standard \texttt{verbatim} environments}

Of course, \begin{verbatim}verbatim may extend 

over multiple\end{verbatim}
lines.
In any event, \verb!\begin{verbatim}! starts a new line and another
new line is added after \verb!\end{verbatim}!.

\verb!\begin{verbatim}! and \verb!\end{verbatim}! may be placed
anywhere in a line.
\verb!\end{verbatim}! may be placed even after a \% sign.


\subsection{\texttt{verbatim} nesting}

verbatim can not be nested.


\subsection{Custom \texttt{verbatim} environments}
There may be custom verbatim environments defined by the \verb!verbatim! package.



\subsection{\texttt{listings} and friends}

For example,
\begin{verbatim}
\begin{lstlisting}[frame=trBL]
for i:=maxint to 0 do
begin
{ do nothing }
end;
\end{lstlisting}
\end{verbatim}



\section{Special Symbols}

\% \$ \# \_ \& \{ \} \P \S \i \j \o \O

\section{Accented Characters}

White space is accepted after the accent command:\newline
\`o \` o \`{o} / 
\'o \' o \'{o} /
\^o \^ o \^{o} /
\"o
\~o
\u o
\=o
\.o
\b{a}
\d o
\t oo
\v o
\H o
\c o
\"{\i}

\noindent
\i 
\j
\o
\O



\section{UTF Character Checks}

\subsection{Basic Latin (U+0000 \ldots U+007F)}
\begin{verbatim*}
 !\"#$%&'()*+,-./0123456789:;<=>?@ABCDEFGHIJKLMNOPQRSTUVWXYZ
[\]^_`abcdefghijklmnopqrstuvwxyz{|}~
\end{verbatim*}
in addition to the control characters
\verb!\n! and \verb!\t!  (we don't accept \verb!\r!).\\
ACUTE ACCENT (U+00B4): not allowed!

%\section{German umlauts}
%\begin{verbatim}
%äöüÄÖÜ
%\end{verbatim}


\subsection{Latin1 supplement (U+0080 \ldots U+00FF)}
\begin{verbatim}
ÀÁÂÃÄÅÆÇÈÉÊËÌÍÎÏÑÒÓÔÕÖØÙÚÛÜÝßàáâãäåæçèéêëìíîïñòóôõöøùúûüýÿ
¡§¿®
\end{verbatim}
%not available: Ð×Þð÷þµ°


\subsection{Latin1 Extended-A (U+0100 \ldots U+017F)}

\begin{verbatim}
ĀāĂăĆćĈĉĊċČčĎďĒēĔĕĖėĚěĜĝĞğĠġĢģĤĥĨĩĪīĬĭİıIJijĴĵĶķĹĺĻļĽľŁł
ŃńŅņŇňŌōŎŏŐőŒœŔŕŖŗŘřŚśŜŝŞşŠšŢţŤťŨũŪūŬŭŮůŰűŴŵŶŷŸŹźŻżŽž
\end{verbatim}
% not: ĄąĐđĘęĦħĮįĸĿŀʼnŊŋŦŧŲųſ


\subsection{Latin1 Extended-B (U+0180 \ldots U+024F)}

\begin{verbatim}
ǍǎǏǐǑǒǓǔǢǣǦǧǨǩǰ
\end{verbatim}


%Beta (β) is not supported.

\section{Quotation marks}

Using “foreign” (compare to ``foreign'') quotation marks does not produce an error but yields
the same results as the original LaTeX marks.



\section{Specific LaTeX Commands}

\begin{itemize}
\item \verb!\break! inserts a line break: Blabla \hfil\break and so on.
\item What does \verb!#! do? -- only allowed as macro parameter number!
\item Can \verb!&! be used outside tables? - No, not allowed!
\end{itemize}


\section{Math stuff}

\subsection{Inline math -- text commands}

\begin{itemize}\itemsep0pt
\item
Inside math, \verb!\text{...}! returns to text mode. $a^b = c, \text{for all time}$.
\item
Same with \verb!\textrm{...}!: $a^b = c, \textrm{for all time}$.
\item
Same with \verb!\textit{...}!: $a^b = c, \textit{for all time}$.
\item
Same with \verb!\textsl{...}!: $a^b = c, \textsl{for all time}$.
\item
Same with \verb!\textsf{...}!: $a^b = c, \textsf{for all time}$.
\item
Same with \verb!\texttt{...}!: $a^b = c, \texttt{for all time}$.
\item
What happens without braces?: $a^b = c, \text for all time $.
\item
Can \verb!\text{...}! be used outside math mode?: \textit{outside math} -- yes!
\item
Can \verb!\mathrm{...}! be used outside math mode?: %\mathrm{outside math} 
-- no!
\item
Does \verb!$\mathrm{...}$! leave math mode?: $\mathrm{a_0 = x^2}$ -- no.
\end{itemize}


\subsection{Inline math -- line breaking commands}

\begin{itemize}\itemsep0pt
\item
Paragraph (\verb!\par!) in math mode? No! %$x = \par y$
\item
\verb!\newline! in math mode? Yes:  $x = \newline y$
\item
\verb!\\! in math mode? Yes:  $x = \\ y$
\end{itemize}



\subsection{Inline math -- verbatim}

\begin{itemize}\itemsep0pt
\item
Can math contain verb? $a^b = c,\ \verb!stuff!\ \text{for all time}$ -- yes!
\item
Can math contain verbatim \emph{environment}?
%$a^b = c,\ \begin{verbatim}stuff\end{verbatim} \text{for all time}$ 
-- no!
\end{itemize}


\end{document}
